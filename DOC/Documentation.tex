\documentclass[a4paper,10pt]{article}

\usepackage{amssymb}
\usepackage{amsmath}

\addtolength{\textwidth}{2.2cm}  
\addtolength{\oddsidemargin}{-1.10cm}
\addtolength{\evensidemargin}{-1.10cm}
\addtolength{\textheight}{2.0cm} 
\addtolength{\topmargin}{-1.2cm}

\usepackage{colortbl}

\usepackage{fancybox}
\usepackage{pgf}
\usepackage{tikz}
\usetikzlibrary{calc,shapes,snakes,arrows,shadows}

\newcommand{\lwaio}[0]{{\tt lwaio}}

\begin{document}

\title{lwaio: a light-weight asynchronous input/output library}

\author{Diego Fabregat-Traver}

\date{December, 2014}

\maketitle

\begin{abstract}
The \lwaio{} library is a light-weight library for asynchronous input/output (I/O).
\lwaio{} replicates the basic functionality of the POSIX {\tt aio} library with
the following additional features: it only spawns one thread to perform the I/O
operations, and this thread may be pinned to a specific core. The main reason
why I developed this library is to overcome the performance penalty observed when
using the original {\tt aio} library to overlap I/O operations
with the execution of highly optimized BLAS~\cite{BLAS3} routines.
This problem was experienced while developing the code presented in~\cite{OmicABEL-SingleTrait,OmicABEL-noMM}.
In the specific case of~\cite{OmicABEL-noMM}, the use of the {\tt aio} library limited scalability of the
code on a 40-core node to about 20x, while with our library we attained more than 36x.
\end{abstract}

\vspace{5mm}

{\Large \bf This document is under construction. If you have any comment or question, feel
free to contact me at
\begin{center} 
    {\tt fabregat@aices.rwth-aachen.de}
\end{center}}

\vspace{5mm}

\section{Overview}

%user space

\section{Interface}

\begin{verbatim}
typedef struct lwaio_task
{
    FILE   *fp;
    off_t  offset;
    size_t nbytes;
    void   *buffer;
} lwaio_task;
\end{verbatim}

\begin{verbatim}
void lwaio_init( void );
void lwaio_finalize( void );

void lwaio_read(  lwaio_task *task );
void lwaio_write( lwaio_task *task );
void lwaio_wait(  lwaio_task *task );

LWAIO_PIN_TO="0"
\end{verbatim}

\section{Internals}

%Single pthread + one task queue (double-linked list) + read/write

%I/O operations are processed in the order they have been enqueued

\section{Example of use: double buffering}

%\bibliography{bibliography}
%\bibliographystyle{plain}
\begin{thebibliography}{1}

\bibitem{BLAS3}
Jack Dongarra, Jeremy~Du Croz, Sven Hammarling, and Iain~S. Duff.
\newblock A set of level 3 basic linear algebra subprograms.
\newblock {\em ACM Trans. Math. Software}, 16(1):1--28, 1990.

\bibitem{OmicABEL-SingleTrait}
Diego Fabregat-Traver, Yurii~S. Aulchenko, and Paolo Bientinesi.
\newblock Solving sequences of generalized least-squares problems on
  multi-threaded architectures.
\newblock {\em Applied Mathematics and Computation (AMC)}, 234:606--617, May
  2014.

\bibitem{OmicABEL-noMM}
Alvaro Frank, Diego Fabregat-Traver, and Paolo Bientinesi.
\newblock Large-scale linear regression: Development of high-performance
  routines.
\newblock {\em Applied Mathematics and Computation}, 275:411--421, February
  2016.

\end{thebibliography}

\end{document}
